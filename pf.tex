\documentclass[14pt,compress,usenames,dvipsnames,aspectratio=169]{beamer}
\useoutertheme{shadow}
\usetheme{CambridgeUS}
\definecolor{mygreen}{RGB}{150, 255, 210}%186}
\definecolor{leftblue}{RGB}{230,255,255}
\definecolor{rightblue}{RGB}{111,195,223}
\definecolor{lefttron}{RGB}{19,44,65}
\definecolor{myblack}{RGB}{27,27,27}
\definecolor{mypurple}{RGB}{205,87,255}

\usecolortheme{owl}

% \setbeamercolor{section in head/foot}{fg = white,bg=black}
\setbeamercolor{title}{fg=green,bg=black}
\setbeamercolor{titlelike}{fg=yellow,bg=black}
\setbeamercolor{item}{fg=green}
\setbeamercolor{block title}{fg=white,bg=myblack!200}
\setbeamercolor{block body}{bg=normal text.bg!80}
\setbeamertemplate{blocks}[rounded][shadow=true]
\setbeamertemplate{headline}{}
\setbeamertemplate{footline}[frame number]
\setbeamercolor{normal text}{fg=white,bg=myblack}%!89.9}

%Gradient
\setbeamercolor{frametitle}{fg=green,bg=black}
\setbeamercolor{frametitle right}{fg=white,bg=gray}

\usepackage[utf8]{inputenc}
\usepackage{amsmath}
\usepackage{amsfonts}
\usepackage{amssymb}
\usepackage{graphicx}
\usepackage{shadowtext}
\usepackage{multicol}
\usepackage[makeroom]{cancel}

\usepackage{natbib}
\usepackage{float}
\usepackage{subcaption}
\usepackage{xcolor}
\usepackage{natbib}
\usepackage{bibentry}
\usepackage{animate}
\usepackage{varwidth}
\usepackage{appendixnumberbeamer}

\usepackage{tikz}
\usetikzlibrary{shapes,arrows}

\title{\textbf{Unterwegs in der Welt der mobilen Betriebssysteme - Ein Reisebericht}}
\author{\textbf{$\varphi$}}
\date{}

\usefonttheme{professionalfonts}

\usepackage{mydefs}

\setbeamercovered{transparent} 
\setbeamertemplate{navigation symbols}{} 

\begin{document}

\setbeamercovered{invisible}

\begin{frame}[plain]
\titlepage
\end{frame}

\begin{frame}{Agenda? oder Reiseplanung?}
    \begin{enumerate}
        \item Station 1: Android
        \item Station 2: Wohin soll die Reise führen?
        \item Obstacel 1: Hardbrick
        \item Obstacle 2: Google Play Dienste
        \item Obstacle 3: Online Banking
        \item "Sie haben ihr Ziel erreicht, das Ziel befindet sich auf der linken Seite"
    \end{enumerate}
\end{frame}

\begin{frame}{Station 1: Android}
    Was ist eigentlich Android? 
    \begin{itemize}
        \item eigentlich ein Open Source Projekt
        \item ein Betriebssysteme für mobile Geräte
        \item “Google decided to give Android away for free and use it as a trojan horse for Google services.”
        \item todo: grafik market share entwicklung ios vs android
    \end{itemize}
\end{frame}

\begin{frame}{Station 1: Android}
    Android in etwas technischer
    \begin{itemize}
        \item auch ein linux
        \item todo: das stack bild
        \item das was wir unter Android verstehen: AOSP mit Properitären Google Anwendungen
        \item kauf ein Gerät bekomme volle Adminrechte. Nicht so bei Android Smartphones.
    \end{itemize}
\end{frame}

\begin{frame}{Station 1: Android}
    Google ist böse oder: warum ich mich auf die Reise gemacht habe.
    \begin{itemize}
        \item Hardware meines Huawei Smartphones war noch super, aber seit zu lange keine Updates mehr 
        \item gegen Monopole: alle beschweren sich immer über Appel, dass die ihr eigenes Universum aufbauen, aber Google ist auch nicht besser
        \item Ich möchte die vollen Rechte auf meinem Gerät haben, die sollen nicht bei Google liegen.
    \end{itemize}
\end{frame}

\begin{frame}{Reiseplanung}
    Was es neben stock android noch gibt
    \begin{itemize}
        \item ios $\to$ keine Option
        \item "freie Androids"
    \end{itemize}
\end{frame}

\begin{frame}{Reiseplanung}
    Was es neben stock android noch gibt
    \begin{itemize}
        \item android family tree todo
    \end{itemize}
\end{frame}

\begin{frame}{Hinderniss 1: Hardbrick}
    oder mein erster Versuch ein Smartphone zu rooten
    \begin{itemize}
        \item da gibt es nicht viel zu sagen, war ungeschickt
    \end{itemize}
\end{frame}

\begin{frame}{Station 2: LineageOS}
    Google play dienste, kaputtes update und was sonst noch so los war
    \begin{itemize}
        \item wie Ubuntu, wenn man sich das erste mal mit Alternativen zu Windoof beschäftigt landet man irgendwie da, so ist das mit LineageOS auch 
        \item keine Google Play Dienste (natürlich) und auch kein microG
        \begin{itemize}
            \item quelloffene, via reverse engineering geschaffene Implementierung der Google Play Dienste 
            \item Framework für vollständig kompatible Android-Distributionen ohne properitäre Google-Komponenten
        \end{itemize}
        \item bisschen Kampf gehabt um alles dann zum Laufen zu bringen und genau dann kommt das erste LineageOS Update und ich hänge beim reboot in der boot loop fest
    \end{itemize}
\end{frame}

\begin{frame}{Hinderniss 2: Google Play Dienste}
    oder warum funktioniert die hälfte nicht mehr
    \begin{itemize}
        \item Gruppe an properitären Hintergrunddiensten und APIs, für Android Geräte (von Google entwickelt)
        \item senden permanet Daten an Google (zum Beispiel Corona-Warn-App)
        \item Maps und logins 
        \item Endgegner: SaftyNet
    \end{itemize}
\end{frame}

\begin{frame}{Station 3: LineageOS for microG}
    \begin{itemize}
        \item kein gefrickel mehr mit microG und so 
        \item relativ Nahe an "Android"
    \end{itemize}
\end{frame}

\begin{frame}{Station 4: /e/ OS}
    \begin{itemize}
        \item /e/ Universum: alles was sonst von G. kommt durch freie Dinge ersetzt (bsp. Nextcould > Google Drive)
    \end{itemize}
\end{frame}

\begin{frame}{Sidequest: Ubuntu touch}
    \begin{itemize}
        \item einfach nein
    \end{itemize}
\end{frame}

\begin{frame}{Station 5: CalyxOS}
    \begin{itemize}
        \item blub
    \end{itemize}
\end{frame}

\begin{frame}{Hinderniss 3: Online Banking}
    oder wie man eine KSK filiale lahmlegt
    \begin{itemize}
        \item blub
    \end{itemize}
\end{frame}

\begin{frame}{"Sie haben ihr Ziel erreicht, das Ziel befindet sich auf der linken Seite"}
    Keine Ahnung wan eine Reise endet oder wann man ein Ziel erreicht hat. 
    Jedenfalls lebt es sich hier (mit CalyxOS) ganz gut.
    Bin jetzt auch am Ende.
\end{frame}
\end{document}